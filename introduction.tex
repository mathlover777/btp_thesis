\section*{Introduction}
In today’s world, online social networks (OSN) are very important media for sharing, organizing
and finding contents and contacts. Analysis of such networks not only helps to reveal many
interesting properties of these networks, but also it helps to improve the present systems and to
design new applications of social networks. The structure of these networks changes from time
to time. Moreover a lot of processes goes on over these networks. To analyze various properties
of such networks, it is important to understand the underlying dynamics and various factors
that drive this dynamics.

Influence between two nodes is an important factor that drive all dynamics in a social net-
work. Therefore in order to understand the dynamics in OSNs, the influence between two nodes
needs to be modeled. However, the influence that a person has on another person significantly
depends on their relationship and the context in which they are. A professor at an university
may not have much influence on who her student connects to, on Facebook. But, the student
may be highly influenced by her professor when it comes to which piece of literature to read
next. Today’s vast proliferation of online social networks (OSN) allows us to study such inter-
actions. Here we present two learning models to address the problems of link prediction and
understanding opinion propagation in OSNs.

The problem of link prediction (LP) is stated as follows: given a graph, for every vertex in
the graph, find vertices to which the given vertex is most likely to form new edges. The problem
of link prediction is very important, in the context of social search and recommendation. Our
method uses two novel signals to improve accuracy in link prediction. First, we use a co-
clustering algorithm to find the underlying communities in the network. We use this information
to qualify edges in the graph with a surprise value. This represents how unexpected the edge is in
the graph, given the underlying community structure information. Second, we compute a node
to node similarity measure which takes into account the local connectivity structure between
these nodes. These signals are then used in combination of others as input to a discriminative
predictor to estimate likelihoods for future edges. When tested across five diverse datasets,
common in link prediction literature, we find our method performs significantly better than
standard link prediction methods.

Today, online social networks constantly bombard users with information. In presence of
such a heavy information overload, it becomes critical to understand which pieces of information
are getting through and which are ignored. A standard approach, is to model OSN users as
actors holding opinions on different topics, which is influenced by their network neighbors.
Here we present a novel linear influence model, which unlike existing approaches, makes no
assumptions about system stability. At a high level, our method tries to predict the opinion of
a user in a social network using a simple model called Voter Model. We have done our experiment on a Twitter and a Reddit dataset. And found that our model works well on the Reddit Dataset. Also we have worked on the voter model and introduced a Kernel function which might give better result in future.